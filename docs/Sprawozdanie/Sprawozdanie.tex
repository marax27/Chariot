\documentclass{article}
\usepackage[a4paper]{geometry}
\usepackage[utf8]{inputenc}
\usepackage{polski}
\usepackage{tabularx}
\usepackage{indentfirst}
\usepackage{multirow}
\usepackage{amssymb}
\usepackage{amsmath}
\usepackage{anysize}
\usepackage{float}
\usepackage{caption}
\usepackage{subcaption}
\usepackage{graphicx}

\usepackage{listings}
\usepackage{color}

\definecolor{mygreen}{rgb}{0,0.6,0}
\definecolor{mygray}{rgb}{0.5,0.5,0.5}
\definecolor{mymauve}{rgb}{0.58,0,0.82}

\usepackage{titling}
\newcommand{\subtitle}[1]{%
	\posttitle{%
	\par\end{center}
	\begin{center}\small#1\end{center}
	\vskip0.5em}%
}

\title{Chariot}
\subtitle{Akademia Górniczo-Hutnicza im. Stanisława Staszica w Krakowie\\
	Wydział Elektrotechniki, Automatyki,\\
	Informatyki i Inżynierii Biomedycznej}
\author{Kacper Tonia\and
		Sławomir Kalandyk}
\date{}

\begin{document}
%------------------------------------------------------------
\maketitle

\section{Cel programu}
Celem programu Chariot jest prosta symulacja dystrybucji wozów strażackich do zgłaszanych pożarów. Przyjęliśmy, że dana jednostka może być w jednym z trzech stanów:
\begin{itemize}
	\item oczekiwanie - jest gotowy do wyjazdu z centrali
	\item w akcji - jest niedostępny, wyjechał z centrali
	\item odpoczynek - powrócił do centrali, musi uzupełnić ekwipunek, załoga musi odpocząć
\end{itemize}

Zgłoszenia są przesyłane za pomocą formularza na stronie HTML.

\section{Struktura programu}

\section{Pakiety zewnętrzne}

\section{Instrukcja obsługi}

\section{Możliwe rozszerzenia programu}

\section{Ograniczenia programu}

\end{document}