\documentclass{article}
\usepackage[a4paper]{geometry}
\usepackage[utf8]{inputenc}
\usepackage{polski}
\usepackage{tabularx}
\usepackage{indentfirst}
\usepackage{multirow}
\usepackage{amssymb}
\usepackage{amsmath}
\usepackage{anysize}
\usepackage{tikz}
\usepackage{float}
\usepackage{caption}
\usepackage{subcaption}
\usepackage{graphicx}

\usepackage{listings}
\usepackage{color}
\usetikzlibrary{shapes}

\definecolor{mygreen}{rgb}{0,0.6,0}
\definecolor{mygray}{rgb}{0.5,0.5,0.5}
\definecolor{mymauve}{rgb}{0.58,0,0.82}

\begin{document}
%------------------------------------------------------------
\section{Moduły}

\begin{itemize}

	\item \textbf{main} -- uruchamia serwer i pełni rolę klienta.

	\item \textbf{central} -- reprezentuje centralę powiadamiania ratunkowego, której można zgłosić żądanie pomocy straży pożarnej. Moduł ten implementuje zachowanie \texttt{gen\_server}.

	\item \textbf{domain} -- definuje rekord \texttt{state} używany przez serwer centrali, a~także tzw.~\emph{invoke functions} -- funkcje realizujące podstawową logikę aplikacji (zapisywanie zgłoszeń, wysyłanie wozów strażackich itd.).

	\item \textbf{firetrucks} -- definiuje rekord \texttt{firetruck} reprezentujący wóz strażacki, a~także podstawowe operacje z~tym rekordem związane.
\end{itemize}

\section{Serwer centrali}
Operacje realizowane przez serwer centrali możemy podzielić na~3~kategorie:
\begin{itemize}
	\item operacje opisane przez \texttt{handle\_call} są~synchroniczne, zwracają pewną wartość nie modyfikując stanu serwera.
	\begin{itemize}
		\item \texttt{get\_vehicles} pobiera aktualną listę pojazdów wraz z~ich stanem
	\end{itemize}

	\item operacje opisane przez \texttt{handle\_cast} są~asynchroniczne, nie blokują klienta. Nie zwracają istotnych wartości, ale mogą modyfikować stan serwera.
	\begin{itemize}
		\item \texttt{report\_incident} zgłasza incydent wymagający przyjazdu straży pożarnej
	\end{itemize}

	\item operacje opisane przez \texttt{handle\_info} sygnalizują serwerowi zajście jakiegoś zdarzenia. Nie powinny być wywoływane z~zewnątrz.
	\begin{itemize}
		\item \texttt{report\_enqueued} sygnalizuje, że~w~kolejce zgłoszeń znajdują się oczekujące zgłoszenia
		\item \texttt{dispatch\_finished} sygnalizuje, że~pojazd wrócił z~akcji strażackiej
		\item \texttt{preparation\_finished} sygnalizuje, że~pojazd jest gotowy do~następnej akcji strażackiej
	\end{itemize}
\end{itemize}

\section{Projekt}
Zadaniem programu jest rozsyłanie wozów strażackich do~incydentów zgłaszanych za~pośrednictwem strony internetowej. Pula wozów jest ograniczona, zaś same pojazdy po~powrocie z~akcji przez pewien czas pozostają niedostępne (czas na~uzupełnienie zapasów, naprawy itd.).

\end{document}